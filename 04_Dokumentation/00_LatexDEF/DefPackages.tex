% 							Package-File - Vorlage für Elektrotechnik
%                    					    2023 HTL Weiz
%----------------------------------------------------------------------------------------------------
% Dieses \LaTeX Template kann als Grundlage für die Erstellung von Diplomarbeitsdokumentationen
% verwendet werden. In diesem File werden sämtliche Pakete eingebunden. 
%----------------------------------------------------------------------------------------------------

%=========================================== P A K E T E ============================================

\usepackage[onehalfspacing]{setspace}																	% Formatierung
\usepackage[usenames, dvipsnames,svgnames,table]{xcolor}												% Tabellen + Farbe
\usepackage[utf8x]{inputenc}																			% Codierung (ASCII, ISO)
\usepackage[ngerman]{babel}																				% Sprache
\usepackage{fancyhdr}																					% Aussehen
\usepackage[T1]{fontenc}																				% Aussehen
\usepackage[scaled]{uarial}																				% Schriftart: Arial (skaliert)
\renewcommand*\familydefault{\sfdefault}																% Standardschriftart
\usepackage[backend=biber, citestyle=numeric, bibstyle=numeric]{biblatex}								% Zitierungen und Literatur
\addbibresource{bibliography.bib}																		% Bib-Datei
\usepackage{hyperref}																					% Quicklink Referenz fuer Literatur
\usepackage{float}																						% Grafiken strukturieren
\usepackage{amsmath}																					% Mathematik
\usepackage{amsfonts}																					% Mathematik
\usepackage{amsthm}																						% Mathematik
\usepackage{amssymb}                    																% Mathematik-Symbole
\usepackage{amsbsy}																						% Mathematik-Symbole
\usepackage{mathrsfs}                   																% Symbole fuer Fourier und Laplace Transformation
\usepackage{calc}																						% Arithmetische Rechenoperationen
\usepackage{eufrak}                     																% Fraktur Symbole
\usepackage{array,longtable,calc,amsmath}																% Tabellen und Mathematik
\usepackage{multirow}																					% Tabellen
\usepackage{longtable}																					% Tabellen
\usepackage{array}																						% Tabellen
\usepackage{tabulary}																					% Tabellen
\usepackage{tabularx}																					% Tabellen
\usepackage{caption}																					% Textformatierung
\usepackage{ifthen}																						% Textformatierung
\usepackage{sidecap}																					% Textformatierung
\usepackage{subcaption}																					% Textformatierung
\usepackage{tikz}																						% Textformatierung
\usepackage{wrapfig}   																					% Textformatierung
\usepackage{titling}																					% Textformatierung
\usepackage{textfit}																					% Textformatierung
\usepackage{chngpage}																					% Textformatierung
\usepackage{lastpage}																					% Textformatierung
\usepackage{setspace}																					% Formatierung
\usepackage{xspace}																						% Formatierung
\usepackage{listings}																					% Codedarstellung
\usepackage{docmute}																					% Datei-Strukturierung
\usepackage{pdfpages}																					% PDF in Latexdokument einfuegen
\usepackage{graphicx}																					% Grafiken
\graphicspath{{./../03_Graphics/}}																		% Grafikenpfad
\usepackage{color}																						% Farben
\usepackage{tcolorbox}																					% Textfelder
\usepackage{scrkbase}																					% Artikelformat
\usepackage{upgreek}																					% Griechische Buchstaben
\usepackage{textgreek}																					% Griechische Buchstaben
\usepackage[printonlyused, withpage]{acronym}															% Akronyme (Abkuerzungen)
\usepackage[acronym,toc]{glossaries}																	% Sprachuebersetzung
\usepackage{enumitem}
\usepackage{adjustbox}
\usepackage{comment}

%============================== M A T H E M A T I K  S Y M B O L E ====================================

\renewcommand{\j}{\ensuremath{\mathrm{j}}} 																% Imaginäere Einheit
\newcommand{\conj}[1]{\ensuremath{{#1}^\ast}} 															% Komplex Konjugiert
\renewcommand{\c}[1]{\ensuremath{\boldsymbol{#1}}} 														% Komplex Fettdruck
%
\newcommand{\E}[1]{\ensuremath{\mathrm{E}\{ {#1} \}}} 													% Expectation E{*}
%
\newcommand{\F}[1]{\ensuremath{\mathcal{F}\{ {#1} \}}} 													% Fourier F{*}
\newcommand{\Fi}[1]{\ensuremath{\mathcal{F}^{-1}\{ {#1} \}}}										 	% Fourier F{*}
%
\renewcommand{\L}[1]{\ensuremath{\mathcal{L}\{ {#1} \}}} 												% Laplace F{*}
\newcommand{\Li}[1]{\ensuremath{\mathcal{L}^{-1}\{ {#1} \}}} 											% Laplace F{*}
%
\newcommand{\e}[1]{\ensuremath{\mathrm{e}^{#1}}} 														% Exponential e^*
%
\renewcommand{\o}[1]{\ensuremath{\overline{#1}}} 														% Overline
%
\newcommand{\abs}[1]{\ensuremath{ | {#1} | }} 															% Absolut |a|
\newcommand{\norm}[1]{\ensuremath{ \| {#1} \| }} 														% Norm ||a||
\newcommand{\avg}[1]{\ensuremath{ < {#1} > }} 															% Zeitintervall <a>
\newcommand{\sign}{\ensuremath{\mathrm{sign}\,}} 														% Signum Funktion
%
\newcommand{\N}[1]{\ensuremath{\mathbb{N}}} 															% Natuerliche Zahlen
\newcommand{\R}[1]{\ensuremath{\mathbb{R}}} 															% Rationale Zahlen
\newcommand{\C}[1]{\ensuremath{\mathbb{C}}} 															% Komplexe Zahlen
\newcommand{\Z}[1]{\ensuremath{\mathbb{Z}}} 															% Ganze Zahlen
\newcommand{\Q}[1]{\ensuremath{\mathbb{Q}}} 															% Irrationale Zahlen
%
\newcommand{\inti}{\ensuremath{\int_{0}^{\,\infty}}} 													% int 0 to infty
\newcommand{\intii}{\ensuremath{\int_{-\infty}^{\,\infty}}} 											% int -infty to infty
\newcommand{\iintii}{\ensuremath{\iint_{-\infty}^{\quad\infty}}} 										% int -infty to infty

