% 							Style-File - Vorlage für Elektrotechnik
%                         				2023 HTL Weiz
%----------------------------------------------------------------------------------------------------
% Dieses \LaTeX Template kann als Grundlage für die Erstellung von Diplomarbeitsdokumentationen
% verwendet werden. In diesem File wird das Aussehen, sowie Designeinstellungen implementiert.
%----------------------------------------------------------------------------------------------------

%****************************************************************************************************
%==================================== S E I T E N F O R M A T =======================================
%****************************************************************************************************

\usepackage[a4paper, left=2.5cm, right=2.5cm, top=2.5cm, bottom=2.5cm]{geometry}						% A4 und Seitenränder
\setlength{\textwidth}{16.0cm}																			% Textbreite
\setlength{\parindent}{0pt}																				% Einzug ausschalten
\setstretch{1.3}																						% Zeilenabstand
\pagestyle{fancy}																						% Seitenformatierung
\setlist[itemize,enumerate]{itemsep=0pt}
\setlength{\parindent}{0pt}
%****************************************************************************************************
%=============================== K O P F-  U N D  F U ß Z E I L E ===================================
%****************************************************************************************************

%======================================== S T A N D A R D ===========================================

% Kopfzeile
\lhead{\LARGE Zero Emission Challenge}						% Baselineskip entfernen falls der Übungsname in eine Zeile passt und das Logo kleiner machen
\rhead{\vspace{-8pt}\includegraphics[scale=0.3]{./00_Introduction/ZeroEmission-Logo.jpg}}
\renewcommand{\headrulewidth}{0.4pt}

% Fußzeile
\lfoot{\iNameOne}
\cfoot{\iYear}
\rfoot{Seite \thepage\ von \pageref{LastPage}}
\renewcommand{\footrulewidth}{0.4pt}

%****************************************************************************************************
%==================================== F A R B E N F O R M A T =======================================
%****************************************************************************************************

\definecolor{kaki}{rgb}{0.74,0.74,0.088}
\definecolor{dred}{rgb}{0.722,0.18,0.18}
\definecolor{darkgrey}{rgb}{0.364,0.411,0.439}
\definecolor{lightgrey}{rgb}{0.776,0.792,0.803}
\definecolor{pythonkeyword}{HTML}{0066CC}
\definecolor{stringstyle}{HTML}{FF5E05}
\definecolor{cppvariable}{HTML}{9CDCFE}
\definecolor{cppkeyword}{HTML}{0066CC}

%****************************************************************************************************
%=============================== G L I E D E R U N G S E B E N E N ==================================
%****************************************************************************************************

% \setcounter{secnumdepth}{5}																			% Gliederungstiefe (hier: 5. Unterüberschrift [1.1.1.1.1]) jedoch eher für "BOOK" geeignet
\setcounter{tocdepth}{2}																				% Inhaltsverzeichnistiefe (hier 2. Unterüberschrift [1.1])

%****************************************************************************************************
%============================== B I L D U N T E R S C H R I F T E N =================================
%****************************************************************************************************

\captionsetup{margin=10pt,font=small, labelfont={color=black!80, bf}, textfont={color=black!60}, format=hang, indention=-1cm}
\captionsetup[wrapfigure]{name=Abbildung}
\captionsetup[figure]{name=Abbildung}

%****************************************************************************************************
%================================= T E X T F E L D V O R L A G E ====================================
%****************************************************************************************************

\newenvironment{Textfeld1}{\begin{tcolorbox}[colback=lightgrey!5,colframe=darkgrey]\small}{\end{tcolorbox}}
\newenvironment{Textfeld2}{\begin{tcolorbox}[colback=lightgrey!5,colframe=blue]\small}{\end{tcolorbox}}

%****************************************************************************************************
%================================== C O D E D A R S T E L U N G =====================================
%****************************************************************************************************

%============================== S T R U K T U R I E R T E R  T E X T ================================

\lstdefinelanguage{Python}{
	keywords={and, as, assert, break, class, continue, def, del, elif, else, except, False, finally, for, from, global, if, import, in, is, lambda, None, nonlocal, not, or, pass, raise, return, True, try, while, with, yield},
	keywordstyle=\color{pythonkeyword},
	commentstyle=\color{green},
	stringstyle=\color{stringstyle},
	sensitive=true,
	comment=[l]{\#},
	morestring=[b]',
	morestring=[b]",
	keywords=[2]{SERVER\_IP, MQTT\_USER, MQTT\_PASSWORD, DB\_USER, DB\_PASSWORD},
	keywordstyle=[2]{\color{pythonkeyword}},
}

\lstdefinelanguage{myCpp}{
	language=C++,
	morekeywords={},
	keywordstyle=\color{cppkeyword},
	commentstyle=\color{green},
	stringstyle=\color{stringstyle},
	sensitive=true,
	comment=[l]{//},
	morecomment=[s]{/*}{*/},
	morestring=[b]",
	morekeywords=[2]{wifi\_ssid, wifi\_password, mqtt\_server, mqtt\_user, mqtt\_password},
	keywordstyle=[2]{\color{cppvariable}}
}

\lstset{
	basicstyle=\small\ttfamily,
	numbers=none,
	rulecolor=\color{black}, 
	numbersep=5pt,
	breaklines=true,
	breakatwhitespace=true,
	tabsize=4,
	columns=flexible,
	frame=none, 
	backgroundcolor=\color{darkgrey!10}
}


